\section{Introduction}
\label{sec:introduction}

Innovation in biomedical deep-learning research is challenging due to sparse data availability, ethical access, and wide contextual variety. Computer vision has the capacity to greatly impact prominent obstacles for biomedicine and support major medical developments. Leveraging image data from brain MRI-scans or cross-sectional pathological microscopies creates a range of challenges such as instance segmentation of malignant tumours or isolating certain nuclei categories and high-threshold detection counts \cite{BBBC039Dataset}. The latter is pertinent for constructing high-throughput monitoring technologies to detect unexpected blood diseases or support biomedical research.

Major limitations in biomedical deep-learning include the availability of low-scale ethically sourced datasets, and flexibly efficient models with high transferability across different learning tasks \cite{BBBC039Dataset,BBBC039ResearchStudy}. By focussing on instance segmentation for various phenotype nuclei provided by the BBBC039 dataset, this study seeks to address two aims: (1) demonstrate efficient tuning and rapid iteration of the U-Net Deep-Layer Aggregation (DLA) model architecture \cite{UNet,DLA}, and (2) evaluate effective segmentation augmentation policies for small datasets. Utilising a small effective segmentation network opens rapid deployment and refinement of deep-learning solutions and suitably benchmarks further research within the field. Tuned augmentation, such as RandAugment, mitigates the issue of low data accessibility and enriches model training effectiveness through regularisation.

The combination of a segmentation-modified RandAugment policy strategy with the U-Net DLA model demonstrated impressive results by outperforming known research for binary nuclei semantic segmentation and presents a new benchmark for 3-class boundary instance segmentation. Efficient loss convergence highlighted the capacity of rapid development, and outperforming existing research indicates the importance of the results from this research \cite{BBBC039ResearchStudy}. The final selected model achieved 3-/2-class test metrics of 90.37/65.39\% accuracy, 97.71/98.77 Dice, 84.37/97.96 AJI, and 90.60/65.98 $F_1$. This improvement indicates a new architecture for multi-task challenges, with proven success for instance segmentation of nuclei.
