\section{Conclusion}

Histopathological nuclei instance segmentation is one of many biomedical research areas. Developing rapidly iterative and effective deep-learning models for image segmentation supports development of multi-task infrastructures and with additional augmentation strategies, mitigates issues of data accessibility across spare research challenges. Deep-Layer Aggregation (DLA) U-Net presents a low-scale model addressing said obstacles, demonstrating success in nuclei segmentation for the BBBC039 dataset. Leveraging a modified RandAugment policy strategy specifically for segmentation highlighted improved results with final binary masking metrics of 98.77 Dice and 97.96 AJI, $+2.337$ and $+7.76$ differences from second-best research results \cite{BBBC039ResearchStudy}. Future work should seek demonstrative multi-task success and improve modified RandAugment search strategies to minimise training iterations and increase the rapid deployment of research in new areas of biomedicine.
